\documentclass{theme}
\usepackage[utf8]{inputenc}
\usepackage[english,swedish]{babel}

\begin{document}
\title{Erwall, a High Level General Purpose Programming Language}
\subtitle{}
\author{Erik Wallström}
\supervisor{Daniel Berg, David Lundholm}
\semester{HT 2017 - VT 2018}
\maketitle
\newpage

\selectlanguage{english}
\begin{abstract}
\end{abstract}

\begin{flushleft}
	{\small {\bf Keywords:} Erwall, Programming, Language, C, Compiler}
\end{flushleft}

\selectlanguage{swedish}
\tableofcontents
\pagenumbering{gobble}
\newpage
\pagenumbering{arabic}

\section{Inledning}

Jag har valt att skapa ett nytt programmeringsspråk som kan användas praktiskt. 
Språket kommer att vara kompilerat samt ha statisk typning. Målet är att 
konstruera en bättre version av det redan existerande programmeringsspråket C. 
Projektet kommer att ha öppen källkod samt en fri upphovsrättslicens. Jag kommer
använda det distribuerade versionshanteringssystemet \textit{git} för att hålla 
ordning på utveckling och ändring av källkod, samt möjliggöra för andra 
människor att bidra med förbättringar och ändringar. Detta är ett ganska 
ambitiöst projekt.

Fokus på programmering och utvecklingen av en kompilator, fokus på språk och 
standard senare. 

Programmering har varit min hobby i nästan 5 år, och under den tiden har jag 
testat ett stort antal av de existerande programmeringsspråken. Dock så har jag 
ännu inte hittat något språk som jag anser vara perfekt; antingen saknas 
funktioner, eller så finns det onödiga och överflödiga funktioner som jag 
ogillar, med mera. Genom att skapa ett eget programmeringsspråk så kan jag 
anpassa och förbättra vissa delar så att det blir det perfekta 
programmeringsspråket för mig. 

\subsection{Bakgrund}

Jag har gått kursen Programmering 1, och går just nu i Programmering 2. Jag har
tidigare erfarenhet av många programmeringsspråk samt \textit{git}, samt skapat 
en interpretator för programmeringsspråket \textit{Brainfuck}. Jag har även 
konstruerat en textredigerare och utvecklingsmiljö för programmeringsspråket
\textit{Lua}, där jag bland annat haft förgmarkeringar för syntax. Språket som
jag har mest erfarenhet av är C, vilket även är mitt favoritspråk och grunden
till mycket inom detta nya språk. Detta gör att jag bedömer att mina kunskaper
är tillräckliga.

Detta projekt kommer fungera som en övning för relativt avancerad programmering,
structurering och arbete av stort projekt. 

\subsection{Syfte}

Syftet med detta projekt är att i slutändan ha ett fungerande 
programmeringsspråk som på sikt kan ersätta programmeringsspråket C, som idag 
fortfarande är ett av de mest använda programmeringsspråket. Dessutom så ska det
visa att ett programmeringsspråk inte är något märkvärdigt; många programmerare
tänker klagar på programmeringsspråk, men ytterst få försöker förbättra dem 
eftersom att mentaliteten att man inte ska återuppfinna det som redan finns, 
samt att man inte ska byta ut det som redan fungerar. Detta leder till att 
programmeringsspråk får en speciell status, och många tänker inte på att det 
rör sig om en viss standard som är satt, eller ett program som helt enkelt 
tolkar och översätter koden du skriver till maskinkod. Många tycker att det är 
orimligt att skapa ett nytt programmeringsspråk.

\subsection{Frågeställningar}

Under projektets gång vill jag försöka utforska och svara på dessa frågor som
jag hade från början. 

\begin{itemize}
	\item Vilka delar är en kompilator uppbyggd av och varför?
		
		Alla stora program är uppbyggda av olika delar som tillsammans utför en 
		uppgift. Så vilka delar är det som en modern kompilator för ett språk
		generellt är uppbyggd av?

	\item Hur bra fungerar C som ett programmeringsspråk för att skapa en 
		kompilator?

		Debatten om vilka språk som passar för vilka uppgifter har hållit på i 
		många år. Språket som jag har mest erfarenhet av än så länge är C, så
		hur bra passar det till just det här projektet? Vilka fördelar och 
		nackdelar finns det då det är väldigt nära hårdvaran, samt att standard-
		biblioteket är minimalt.

	\item Är C ett bra mål att generera kod till?

		Är det en bra idé att översätta koden skriven i det nya språket till C
		kod istället för till exemepl maskinkod eller använda virtuella maskiner
		som till exempel \textit{LLVM}? Vilka fördelar och nackdelar kan det ha?

	\item Vad krävs av ett programmeringsspråk för att det ska anses som "bra"?

		Vad är det egentligen som gör att folk gillar och ogillar språk? Var 
		går balansen mellan bra och användbar? Skaparen av programmeringsspråket
		C++ har bland annat sagt ``Det finns två olika typer av språk; språk som
		folk klagar på, och språk som inte används``. 

	\item Hur ska man planera ett projekt för att tillåta enkel felsökning och 
		tillägg av nya funktioner?

		Stora projekt så som detta kommer vara kommer kräva många ändringar och
		försvåra felsökning och förbättringar. Vilka verktyg och praxis bör man
		använda och följa för att förenkla detta?

\end{itemize}

\subsection{Kravspecifikation}

\begin{itemize}
	\item Logisk och konsekvent syntax
	\item Explicita deklarationer
	\item Högpresterande och nära maskinvaran
\end{itemize}

Några specifika specifikationer på språket kommer vara följande: 

\begin{itemize}
	\item Rekursiva omfång (?) med funktioner och kommentarer
	\item Garanterade svansanrop (?)
	\item Kompatibelt med C 
	\item Strikt typsystem med riktiga konstanter
	\item Inget odefinerat beteende
	\item Listor som första klassens medborgare
	\item Markerade unioner

	med mera.
\end{itemize}

\subsection{Metod och material}

Projektet bygger på research utifrån tekniska artiklar och wikipedia mm. 
Projektet är skrivet i ren gnu-11 C kod med hjälp av ett textredigeringsprogram 
(vim) i operativsystemet Arch Linux och kompilerat med kompilatorn GCC. Inga 
externa bibliotek och API:er har använts, utan allt är skrivet från grunden. För
versionhantering har git använts, och för debugging har gdb och valgrind 
använts. 

Motivera....

\subsection{Teoretisk bakgrund}
This subsection's content...

\section{Resultatredovisning}

\subsection{Arbetsgången}

Under de första veckorna var det endast planering, research om ämnet, 
brainstorming av idéer samt design av syntax som gällde. \\

\noindent\textbf{2017-10-17}\\
\noindent\rule{\textwidth}{1pt}

\noindent 
Första prototypen av en lexer samt alla grundläggande strukturer blev klara 
idag. Eftersom att standard-biblioteket i C saknar mycket behövde jag 
implementera bland annat dynamiska generiska listor, hantering av färgkoder
för Linux-terminalen samt en abstraktion av filhanteringgränssnittet. Lexern 
har enkel felrapportering där både rad och kolumn där felet hittades visas, 
operatorer, nyckelord, identifierare samt sträng-, boolean- och 
nummerlitteraler, mångradiga kommentarer som kan vara nested. \\

\noindent\textbf{2017-10-19}\\
\noindent\rule{\textwidth}{1pt}

\noindent 
Idag blev första parser-prototypen färdig. Jag skrev ett gränssnitt för 
abstrakta syntaxträd, och lade till logisk utmatning så att man enkelt och 
grafiskt kan se hur trädet är uppbyggt. Parsern är en recursive descent parser, 
som använder tokens från lexern samt AST-gränsittet för att konstruera ett träd.
Just nu stödjer det bland annat hantering av funktioner och block, med logiska 
felrapporter. Jag upptäckte att syntaxen för flerradiga kommentarer orsakade 
problem i vissa specifika fall, och ändrade därför i lexern för att lösa detta.
\\

\noindent\textbf{2017-10-20}\\
\noindent\rule{\textwidth}{1pt}

\noindent 
Parsern stödjer nu numeriska uttryck med alla vanliga matematiska operatorerna,
funktionsanrop samt variabeldeklarationer. Lexern stödjer nu även 
bitwise-operatorer.\\

\noindent\textbf{2017-10-22}\\
\noindent\rule{\textwidth}{1pt}

\noindent 
Nu stödjer parsern även if-satser och booleanska uttryck.\\

\noindent\textbf{2017-10-23}\\
\noindent\rule{\textwidth}{1pt}

\noindent 
Idag började jag arbeta på semantisk analys av AST:t. Kompilatorn rapporterar 
nu om odefinerade variabler, funktioner och typer, samt när du anropar en 
funktion med inkorrekt antal argument med mera.  Jag skapade ett även ett 
gränssnitt för scopes, fixade så att parsern nu kan hantera typdeklarationer och
return statements, och började komma på idéer om syntax för listor samt 
referenser för språket. \\

\noindent\textbf{2017-10-28}\\
\noindent\rule{\textwidth}{1pt}

\noindent 
Jag har nu lyckats lägga till tester för logiska och numeriska operationer och
funktionsdeklarationer. Parsern stödjer nu även typomvandlingar.\\

\noindent\textbf{2017-10-29}\\
\noindent\rule{\textwidth}{1pt}

\noindent 
Idag lade jag till så att parsern klarar av elseif och else satser, samt externa
funktionsanrop. Dessutom lade jag till semantiska kontroller för 
main-funktionen.\\

\noindent\textbf{2017-11-01}\\
\noindent\rule{\textwidth}{1pt}

\noindent 
Nu har jag lagt till experimentell kodgenerering. Jag var tvungen att själv 
skapa en abstraktion av C-stränggränssnittet eftersom att det inte var 
tillräckligt användbart för mina ändamål. \\

\noindent\textbf{2017-11-02}\\
\noindent\rule{\textwidth}{1pt}

\noindent 
Kompilatorn stödjer nu argument för att stödja frivillig utskrivning av AST och 
tokens med mera. C har inget sådant bibliotek i standarden, så jag fick skriva 
en API för POSIX-baserade kommandoradsargument.\\

\noindent\textbf{2017-11-03}\\
\noindent\rule{\textwidth}{1pt}

\noindent 
Kompilatorn stödjer nu automatisk kompilering av den genererade C koden med 
hjälp av GCC. \\

\noindent\textbf{2017-11-04}\\
\noindent\rule{\textwidth}{1pt}

\noindent 
Parsern och kodgeneratorn stödjer nu tomma typer. Generatorn skapar nu korrekt 
indenterad C kod, och genererar if-satser korrekt. Semantiska tester för externa
funktionsanrop samt memory leaks i semantiska tester är nu fixade. \\

\noindent\textbf{2017-11-05}\\
\noindent\rule{\textwidth}{1pt}

\noindent 
Små förbättringar av funktionsgenereringen lades till.\\

\noindent\textbf{2017-11-07}\\
\noindent\rule{\textwidth}{1pt}

\noindent 
Idag lade jag till generering av uttryck samt extra semantisk information in i 
AST noderna. Experimentell generering av lokala funktioner och funktionsanrop 
lades även till. \\

\noindent\textbf{2017-11-08}\\
\noindent\rule{\textwidth}{1pt}

\noindent 
Idag fixade jag en bug som gjorde att lokala variabler inte genererades korrekt.
\\

\noindent\textbf{2017-11-26}\\
\noindent\rule{\textwidth}{1pt}

\noindent 
Jag har nu arbetat i mer än två veckor med att omstrukturera hela 
projektet.  Anledningen var att felrapporteringen inte var tillräckligt bra, 
samt att mycket blev rörigt. Det var bättre att strukurera om helt än att ändra 
på allt som redan var skrivet bedömde jag. Hela projektet är nu betydligt bättre
strukturerat vilket gör att ändringar blir enklare. Felrapporteringen är nu 
tydlig och färgkodad. Semantiska analyser för oanvända och oinitierade variabler
fungerar nu korrekt, och parsern stödjer nu tomma return-statements. \\

\noindent\textbf{2017-12-01}\\
\noindent\rule{\textwidth}{1pt}

\noindent 
Semantisk analys för att se om funktioner returnerar korrekt fungerar nu. Jag 
fixade även lite memory leaks som jag upptäckte. Jag började även med 
enkel optimering, men det visade sig vara betydligt svårare än jag trodde.\\

\noindent\textbf{2017-12-21}\\
\noindent\rule{\textwidth}{1pt}

\noindent 
Under veckan har jag testat att implementera en enkelt interpretator för 
compile-time execution av kod. Efter ett tag bedömde jag dock att detta var
ett för stort ämne att börja på nu. \\

\noindent\textbf{2017-12-29}\\
\noindent\rule{\textwidth}{1pt}

\noindent 
Jag har lagt till små förbättringar till generator och testerna. Bland annat 
klarar kodgeneratorn nu externa funktionsanrop, if-satser samt typkonversioner. 
\\

\noindent\textbf{2018-01-01}\\
\noindent\rule{\textwidth}{1pt}

\noindent 
Återigen har jag fixad kodgeneratorn då den fortfarande hade vissa problem. 
Lokala funktioner samt exponenter i uttryck fungerar nu som de ska göra.\\

\noindent\textbf{2018-01-05}\\
\noindent\rule{\textwidth}{1pt}

\noindent 
Idag gjorde jag så att parsern och generatorn hanterar defer-statements. \\

\noindent\textbf{2018-01-06}\\
\noindent\rule{\textwidth}{1pt}

\noindent 
While-loopar parse:as och generaras nu korrekt. Jag lade också till 
förbättringar till felmeddelanden.\\

\noindent\textbf{2018-01-20}\\
\noindent\rule{\textwidth}{1pt}

\noindent 
Prestanda-tester har nu lagds till, så att man enkelt kan se hur lång tid varje
komponent i kompilatorn tar. Fixade även några små fel som jag hittade.\\

\noindent\textbf{2018-01-21}\\
\noindent\rule{\textwidth}{1pt}

\noindent 
Ett riktigt typsystem har nu blivit implementerat. Det kan hantera listor, 
pekare, funktioner, unioner, enumerationer structurer samt typdefinitioner.\\

\subsection{Resultatet}

Kompilatorn och språket är långt ifrån klart. Många av specifikationerna som 
sattes i början av projektet han inte bli färdiga. Dock så är Erwall nu ett 
fungerande programmeringsspråk som jag bland annat använde en matematiklektion,
vilket tyder på att det redan är användbart och på vissa sätt till och med 
bättre än C. Bland annat är typsystemet betydligt mer avancerat, och konstanter
samt sematiska kontroller och felmeddelande är betydligt bättre i jämförelse. I 
teorin skulle man kunna skriva vilka program som helst i det. 

\section{Diskussion och slutsatser}



\section{Källförteckning}
This section's content...

\section{Bilagor}
This section's content...

\end{document}
