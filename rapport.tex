\documentclass{theme}
\usepackage[utf8]{inputenc}
\usepackage[english,swedish]{babel}

\begin{document}
\title{Erwall, a High Level General Purpose Programming Language}
\subtitle{}
\author{Erik Wallström}
\supervisor{Daniel Berg, David Lundholm}
\semester{HT 2017 - VT 2018}
\maketitle
\newpage

\selectlanguage{english}
\begin{abstract}
\end{abstract}

\begin{flushleft}
	{\small {\bf Keywords:} Erwall, Programming, Language, C, Compiler}
\end{flushleft}

\selectlanguage{swedish}
\tableofcontents
\pagenumbering{gobble}
\newpage
\pagenumbering{arabic}

\section{Inledning}

Jag har valt att skapa ett nytt programmeringsspråk som kan användas praktiskt. 
Språket kommer att vara kompilerat samt ha statisk typning. Målet är att 
konstruera en bättre version av det redan existerande programmeringsspråket C. 
Projektet kommer att ha öppen källkod samt en fri upphovsrättslicens. Jag kommer
använda det distribuerade versionshanteringssystemet \textit{git} för att hålla 
ordning på utveckling och ändring av källkod, samt möjliggöra för andra 
människor att bidra med förbättringar och ändringar. Detta är ett ganska 
ambitiöst projekt.

Fokus på programmering och utvecklingen av en kompilator, fokus på språk och 
standard senare. 

Programmering har varit min hobby i nästan 5 år, och under den tiden har jag 
testat ett stort antal av de existerande programmeringsspråken. Dock så har jag 
ännu inte hittat något språk som jag anser vara perfekt; antingen saknas 
funktioner, eller så finns det onödiga och överflödiga funktioner som jag 
ogillar, med mera. Genom att skapa ett eget programmeringsspråk så kan jag 
anpassa och förbättra vissa delar så att det blir det perfekta 
programmeringsspråket för mig. 


\subsection{Bakgrund}

Jag har gått kursen Programmering 1, och går just nu i Programmering 2. Jag har
tidigare erfarenhet av många programmeringsspråk samt \textit{git}, samt skapat 
en interpretator för programmeringsspråket \textit{Brainfuck}. Jag har även 
konstruerat en textredigerare och utvecklingsmiljö för programmeringsspråket
\textit{Lua}, där jag bland annat haft förgmarkeringar för syntax. 

\subsection{Syfte}

Syftet med detta projekt är att i slutändan ha ett fungerande 
programmeringsspråk som på sikt kan ersätta programmeringsspråket C, som idag 
fortfarande är ett av de mest använda programmeringsspråket. Dessutom så ska det
visa att ett programmeringsspråk inte är något märkvärdigt; många programmerare
tänker klagar på programmeringsspråk, men ytterst få försöker förbättra dem 
eftersom att mentaliteten att man inte ska återuppfinna det som redan finns, 
samt att man inte ska byta ut det som redan fungerar. Detta leder till att 
programmeringsspråk får en speciell status, och många tänker inte på att det 
rör sig om en viss standard som är satt, eller ett program som helt enkelt 
tolkar och översätter koden du skriver till maskinkod. Många tycker att det är 
orimligt att skapa ett nytt programmeringsspråk.

\subsection{Frågeställningar}

Under projektets gång vill jag försöka utforska och svara på dessa frågor som
jag hade från början. 

\begin{itemize}
	\item Vilka delar är en kompilator uppbyggd av?
		
		Alla stora program är uppbyggda av olika delar som tillsammans utför en 
		uppgift. Så vilka delar är det som en modern kompilator för ett språk
		generellt är uppbyggd av?

	\item Hur bra fungerar C som ett programmeringsspråk för att skapa en 
		kompilator?

		Debatten om vilka språk som passar för vilka uppgifter har hållit på i 
		många år. Språket som jag har mest erfarenhet av än så länge är C, så
		hur bra passar det till just det här projektet? Vilka fördelar och 
		nackdelar finns det då det är väldigt nära hårdvaran, samt att standard-
		biblioteket är minimalt.

	\item Är C ett bra mål att generera kod till?

		Är det en bra idé att översätta koden skriven i det nya språket till C
		kod istället för till exemepl maskinkod eller använda virtuella maskiner
		som till exempel \textit{LLVM}? Vilka fördelar och nackdelar kan det ha?

	\item Vad krävs av ett programmeringsspråk för att det ska anses som "bra"?

		Vad är det egentligen som gör att folk gillar och ogillar språk? Var 
		går balansen mellan bra och användbar? Skaparen av programmeringsspråket
		C++ har bland annat sagt ``Det finns två olika typer av språk; språk som
		folk klagar på, och språk som inte används``. 

	\item Hur ska man planera ett projekt för att tillåta enkel felsökning och 
		tillägg av nya funktioner?

		Stora projekt så som detta kommer vara kommer kräva många ändringar och
		försvåra felsökning och förbättringar. Vilka verktyg och praxis bör man
		använda och följa för att förenkla detta?

\end{itemize}

\subsection{Kravspecifikation}

\begin{itemize}
	\item Logisk och konsekvent syntax
	\item Explicita deklarationer
	\item Högpresterande och nära maskinvaran
\end{itemize}

Några specifika specifikationer på språket kommer vara följande: 

\begin{itemize}
	\item Rekursiva omfång (?) med funktioner och kommentarer
	\item Garanterade svansanrop (?)
	\item Kompatibelt med C 
	\item Strikt typsystem med riktiga konstanter
	\item Inget odefinerat beteende
	\item Listor som första klassens medborgare
	\item Markerade unioner

	med mera.
\end{itemize}

\subsection{Metod och material}

Projektet är skrivet i ren gnu-11 C kod med hjälp av ett textredigeringsprogram 
(vim) i operativsystemet Arch Linux och kompilerat med kompilatorn GCC. Inga 
externa bibliotek och API:er har använts, utan allt är skrivet från grunden. För
versionhantering har git använts, och för debugging har gdb och valgrind 
använts.

Motivera....

\subsection{Teoretisk bakgrund}
This subsection's content...

\section{Resultatredovisning}
This section's content...

\subsection{}

\section{Diskussion och slutsatser}
This section's content...

\section{Källförteckning}
This section's content...

\section{Bilagor}
This section's content...

\end{document}
