\documentclass{theme}
\usepackage[utf8]{inputenc}
\usepackage[english,swedish]{babel}

\begin{document}
\title{Erwall, a High Level General Purpose Programming Language}
\subtitle{}
\author{Erik Wallström}
\supervisor{Daniel Berg, David Lundholm}
\semester{HT 2017 - VT 2018}
\maketitle
\newpage

\selectlanguage{english}
\begin{abstract}
	The primary objective of this project is to implement a fully working 
	imperative compiled high performance general purpose programming language in
	C, from scratch without using any external libraries or application
	programming interfaces. The language will be offering many modern features
	with a consistent syntax, so that it can ultimately replace 
	the C language in practice. Many commonly used tools will be used, such as
	git, gcc, valgrind and gdb. I will go through the general compiler pipeline 
	and implement each component step by step. I was not able to reach all the 
	goals that was planned in the beginning, but the project was far 
	from a failure. The final result was a functioning, turing complete 
	programming language that could used for any computing task in theory. It 
	already offers many improvements over C, and I belive that it will become a
	complete replacement for C in the near future. The project would probably 
	have become far more mature if I used existing tools such as a parser and 
	lexer generator. A significant amount of time was used to implement 
	interfaces that was missing in the C standard library, which means that I
	probably would have had the time to implement more features for Erwall if
	I used a scripting language such as python. 
\end{abstract}

\begin{flushleft}
	{\small {\bf Keywords:} Erwall, Programming, Language, C, Compiler}
\end{flushleft}

\selectlanguage{swedish}
\tableofcontents
\pagenumbering{gobble}
\newpage
\pagenumbering{arabic}

\noindent \large \textbf{OBS: Första utkast! Många stavfel, grammatikfel och 
mycket innehåll saknas!}\\
\section{Inledning}

\normalsize

Dagens informationssamhälle är helt beroende av datorer. Fler och fler
jobb automatiseras, och behovet att kunna skapa och utveckla felsäkra program 
som ska kunna köras på dem är större än någonsin. För att skapa ett 
datorprogram måste man på något sätt kunna kommunicera med datorer för att 
framföra vad man vill göra. Datorer förstår dock endast maskinkod; ettor och
nollor. Man insåg redan på 1950-talet att det var ohållbart för människor att
skriva maskinkod för hand, eftersom att det var oerhört tidskrävande och ledde
ofta till fel. Därför skapade man så kallade programmeringsspråk; kod som
består av text och siffror som är betydligt lättare för människor att hantera.
Under åren har det skapats väldigt många olika programmeringsspråk. 

Man kan dela upp programmeringsspråk i flera kategorier, bland annat dynamiska
och statiska, generella och specifika, imperativa och deklarativa med mera. Ett 
statiskt programmeringsspråk översätts till maskinkod med hjälp av ett program
som kallas för en kompilator. Dynamiska språk använder sig istället av interpretatorer, som 
är som virtuella maskiner som kör koden. Generella programmeringsspråk är språk 
som kan användas till vad som helst, till exempel datorspel, 
textredigeringsprogram eller webbläsare. Specifika programmeringsspråk är 
endast avsedda för en enda uppgift, till exempel statistik eller 
specialeffekter. Om du använder ett imperativt programmeringsspråk måste du 
steg för steg förklara för datorn hur den ska göra för att lösa en uppgift, 
medan du endast behöver säga vad du vill få löst i ett deklarativt språk. Man 
paratar också ofta om hög- och lågnivåspråk. Ett lågnivåspråk är väldigt lik 
maskinkod, och kan endast göra det som processorn kan göra, medan högnivåspråk 
har många abstraktioner som gör att man kan ha variabler, funktioner och 
procedurer som en processor inte direkt kan förstå.

\subsection{Bakgrund}

Programmering har varit min hobby i nästan 5 år, och under den tiden har jag 
testat ett stort antal av de existerande programmeringsspråken. Dock så har jag 
ännu inte hittat något språk som jag anser vara perfekt; antingen saknas 
funktioner, eller så finns det onödiga och överflödiga funktioner som jag 
ogillar, med mera. Därför har jag valt att skapa ett nytt imperativt statiskt
högnivåprogrammeringsspråk som kan användas i praktiken. Genom att skapa ett 
eget programmeringsspråk så kan jag anpassa och förbättra vissa delar så att det
blir det perfekta programmeringsspråket för mig.  Projektet kommer att ha öppen 
källkod samt en fri upphovsrättslicens. 

Jag har gått kursen Programmering 1, och går just nu i Programmering 2. 
Tidigare har jag skapat en interpretator för programmeringsspråket 
\textit{Brainfuck}, en textredigerare och utvecklingsmiljö med syntax 
highligting för programmeringsspråket \textit{Lua}. Jag har även
erfarenhet av många programmeringsspråk, inklusive men inte begränsat till
\textit{Java}, \textit{C}, \textit{Lua}, \textit{C++} och \textit{Ruby}. Allt 
detta gör att jag bedömer att jag är tillräckligt kompetent för att klara detta 
projekt, även om det är väldigt ambitiöst för en person. 

Mycket av detta nya språk kommer att bygga på programmeringsspråket C, som är 
mitt favoritspråk och det språk som jag har mest erfarenhet av. C skapades på
1970-talet och är ett statiskt högpresterande imperativt högnivåspråk, och är 
idag ett av det mest använda programmeringsspråken. Många andra 
programmeringsspråk är skapade i C, och det finns mycket resurser och 
information om det eftersom att det är så pass populärt. Detta gör C till ett 
väldigt bra språkval för just detta projektet. 

\subsection{Syfte}

Syftet med detta projekt är att få fram ett fungerande programmeringsspråk som
på sikt kan ersätta programmeringsspråket C. Språket ska ha en del moderna 
egenskaper, göra det enklare för programmerare att skapa högpresterande
och felsäkra program, samt göra det roligare att programmera. 

Detta projekt ska dessutom visa att det inte är så märkvärdigt att skapa ett 
programmeringsspråk; många programmerare klagar på programmeringsspråk, men 
ytterst få försöker förbättra dem. Programmeringsspråk brukar ha en speciell 
status, speciellt bland nybörjare, på grund av mentaliteten att man inte ska 
återuppfinna det som redan finns, att det är för svårt, samt att man inte ska 
byta ut det som redan fungerar. Många tycker därför att det är orimligt att 
skapa ett nytt programmeringsspråk, och tänker inte på att det bara rör sig om 
en viss standard som är satt, eller att en kompilator är ett program som helt 
enkelt tolkar och översätter koden du skriver till maskinkod. 

Detta projekt kommer även fungera som en övning för att träna på avancerad 
programmering, strukturering och arbete av stort projekt och göra en till en 
bättre programmerare. För att vara en bra programmerare måste man förstå hur 
programmeringsspråk är uppbyggda och fungerar, vilket gör att detta projekt är 
perfekt för ändamålet.

\subsection{Frågeställningar}

\begin{itemize}
	\item Vilka delar är en kompilator uppbyggd av och varför?

	\item Hur bra fungerar C som ett programmeringsspråk för att skapa en 
		kompilator?

	\item Är C ett bra mål att generera kod till?

	\item Hur ska man planera ett projekt för att tillåta enkel felsökning och 
		tillägg av nya funktioner?

\end{itemize}

\subsection{Kravspecifikation}

\textbf{Generella krav:}

\begin{itemize}
	\item Logisk och konsekvent syntax

		Många språk har ologisk och inkonsekvent form, där till exempel
		funktionsdeklarationer inte ser likadana ut beroende på vart de
		är deklarerade, och vissa variabeldeklarationer ser ut som matematiska
		uttryck. Funktionsargument ser generellt sett inte ut som variabler, 
		fastän de i praktiken är exakt likadana. Detta gör det jobbigt för 
		programmerare, och därför ska sådana specialfall inte existera i Erwall.

	\item Explicita deklarationer

		När du låter kompilatorn gissa vad du vill att den ska göra kallas det 
		för implicita deklarationer. Då skriver du inte ut exakt vad du vill att
		den ska göra. Detta gör att det kan bli svårare att förstå koden, och 
		kan leda till logiska fel. Därför ska Erwall endast ha explicita 
		deklarationer.

	\item Högpresterande

		Vissa programmeringsspråk prioriterar hastigheten som man kan skapa 
		program med framför hur snabbt programmet körs. Man kan tänka att man
		föredrar kvantitet före kvalitet. Erwall kommer dock att prioritera 
		hastigheten som programmet körs med, så att kvalitén på produkten 
		programmeraren skapar blir så bra som den kan bli.\\

\end{itemize}

\noindent \textbf{Några specifika specifikationer på språket kommer vara 
följande:}

\begin{itemize}
	\item Stöd för \textit{nested} kommentarer och funktioner

		I många programmeringsspråk kan du inte ha kommentarer i kommentarer, 
		eller funktioner i funktioner. Detta är ologisk, och kan i många fall
		göra det jobbigare för programmerare. 

	\item Garanterade tail-call optimeringar

		Vissa språk så som C har inte garanterade tail-call optimeringer, som 
		ofta krävs för att programmera på ett funktionellt sätt; programmet 
		kommer att krasha efter ett tag om man inte har det. Erwall kommer 
		att stödja detta.

	\item Kompatibelt med C 

		Eftersom att C är ett så populärt och moget språk finns det väldigt 
		många bibliotek och gränssnitt skrivna i C. Genom att göra Erwall 
		kompatibelt med C kan man använda dessa redan skapade bibliotek istället
		för att skapa allt från grunden, vilket är orimligt.

	\item Strikt typsystem med riktiga konstanter

		Många språk har inte strikta typsystem. Detta gör att till exempel 
		heltal implicit kan omvandlas till decimaltal. Detta kommer inte vara 
		tillåtet i Erwall.

	\item Inget odefinerat beteende

		I till exempel C kan man skriva kod som har odefinerat beteende; ingen 
		vet vad som kommer hända. Detta gör att det blir lätt att skriva 
		felaktig kod. I Erwall kommer all kod som kompileras vara korrekt och 
		definerad, för att underlätta för debugging som ofta tar väldigt lång 
		tid.

	\item Listor som första klassens medborgare

		En entitet som stödjer vanliga operationer, så som att de kan fungera 
		som argument, returneras och tilldelas till variabler, kallas för första
		klassens medborgare. I många språk är listor inte första klassens
		medborgare, vilket gör att de sällan används. Istället används bibliotek
		och andra gränssnitt. Erwall kommer att ha stöd för dem, så att man 
		slipper detta. Listor behövs för nästan alla projekt, och bör vara 
		inbyggda i språket. 

	\item Tagged unions

		En union är en datatyp som kan ha en av flera definerade typer. I C
		vet man inte vilken av typerna unionen har. Detta gör att man själv
		måste hålla koll på det. Erwall kommer att ha detta inbyggt, så att 
		man själv slipper göra det.

\end{itemize}

\subsection{Metod och material}

Fokus lades på att skapa en fungerande kompilator. Språket kommer att anpassas
och utvecklas efter hand. Detta gjordes då ett av målen var att språket ska
kunna användas i praktiken. Hade det varit ett mer teoretiskt arbete skulle 
fokus istället läggas på språk
specifikationer och en standard. Projektet bygger
på research utifrån tekniska artiklar, föreläsningar samt wikipedia. Utifrån
informationen som samlades ihop skapades en projektplan som sedan följdes. Genom
detta kunde frågeställningen ``vilka delar är en kompilator uppbyggd av`` besvaras.

En \textit{lexer} skapades först. Detta är den första komponenten i de
flesta kompilatorer. Denna komponent delar upp koden i så kallade \textit{tokens},
som används för att särskilja speciella kodord med identifierare, som är namn på
variabler och funktioner som programmeraren kan definera. Därefter implementerades
en \textit{parser}, som konstruerar ett abstrakt syntaxträd (AST) av \textit{tokens}, så att man tydligt
kan se hierarki och gruppereringar. Efter det skapades en semantisk analysator, för
att kontrollera om koden programmeraren har skrivet är logisk, till exempel om man
verkligen kan addera text med nummer. Den semantiska analysator konstruerar även
en symboltabell. Sedan skapades en kodgenerator, som tar trädet och symboltabellen 
och genererar C kod. Till sist skapades ett typsystem. Vid implementationen av 
kodgeneratorn kunde svaret till frågan om C är ett bra mål att generera kod till
besvaras.

Alla komponenter skrevs på en dator med Arch Linux i programmeringsspråket C 
(gnu-c11) med hjälp av textredigeringsprogrammet \textit{vim} och kompilatorn
\textit{gcc}, för att kunna besvara frågan om C fungerar bra för att skapa en
kompilator. För felsökning och kontroller användes verktygen \textit{gdb} och
\textit{valgrind}, som är standard för debugging av C program i Linux. Inga externa
bibliotek eller \textit{application programming interfaces} (API)  har använts; allt är skrivet från grunden. Detta 
gjordes så att man verkligen förstår processen. Om färdigskrivna gränssnitt hade 
använts hade man inte kunnat anpassa det lika bra, och mycket av implementationen
hade då varit färdiggjort och på så sätt hade det varit svårt att besvara många av
frågeställningarna.

Under projektets gång användes det distribuerade versionhanteringssystemet 
\textit{git} för att hålla ordning på utveckling, ändring av kod och för att 
förenkla felsökning. Dessutom möjliggjorde detta för andra människor att bidra
med förbättringar och ändringar, då Erwalls kompilator har öppen källkod. 

\subsection{Teoretisk bakgrund}
This subsection's content...

\section{Resultatredovisning}

\subsection{Arbetsgången}

Under de första veckorna var det endast planering, research om ämnet, 
brainstorming av idéer samt design av syntax som gällde. \\

\noindent\textbf{2017-10-17}\\
\noindent\rule{\textwidth}{1pt}

\noindent 
Första prototypen av en lexer samt alla grundläggande strukturer blev klara 
idag. Eftersom att standard-biblioteket i C saknar mycket behövde jag 
implementera bland annat dynamiska generiska listor, hantering av färgkoder
för Linux-terminalen samt en abstraktion av filhanteringgränssnittet. Lexern 
har enkel felrapportering där både rad och kolumn där felet hittades visas, 
operatorer, nyckelord, identifierare samt sträng-, boolean- och 
nummerlitteraler, mångradiga kommentarer som kan vara nested. \\

\noindent\textbf{2017-10-19}\\
\noindent\rule{\textwidth}{1pt}

\noindent 
Idag blev första parser-prototypen färdig. Jag skrev ett gränssnitt för 
abstrakta syntaxträd, och lade till logisk utmatning så att man enkelt och 
grafiskt kan se hur trädet är uppbyggt. Parsern är en recursive descent parser, 
som använder tokens från lexern samt AST-gränsittet för att konstruera ett träd.
Just nu stödjer det bland annat hantering av funktioner och block, med logiska 
felrapporter. Jag upptäckte att syntaxen för flerradiga kommentarer orsakade 
problem i vissa specifika fall, och ändrade därför i lexern för att lösa detta.
\\

\noindent\textbf{2017-10-20}\\
\noindent\rule{\textwidth}{1pt}

\noindent 
\textit{Parser}n stödjer nu numeriska uttryck med alla vanliga matematiska operatorerna,
funktionsanrop samt variabeldeklarationer. Lexern stödjer nu även 
bitwise-operatorer.\\

\noindent\textbf{2017-10-22}\\
\noindent\rule{\textwidth}{1pt}

\noindent 
Nu stödjer \textit{parser}n även if-satser och booleanska uttryck.\\

\noindent\textbf{2017-10-23}\\
\noindent\rule{\textwidth}{1pt}

\noindent 
Idag började jag arbeta på semantisk analys av AST:t. Kompilatorn rapporterar 
nu om odefinerade variabler, funktioner och typer, samt när du anropar en 
funktion med inkorrekt antal argument med mera.  Jag skapade ett även ett 
gränssnitt för scopes, fixade så att parsern nu kan hantera typdeklarationer och
return statements, och började komma på idéer om syntax för listor samt 
referenser för språket. \\

\noindent\textbf{2017-10-28}\\
\noindent\rule{\textwidth}{1pt}

\noindent 
Jag har nu lyckats lägga till tester för logiska och numeriska operationer och
funktionsdeklarationer. Parsern stödjer nu även typomvandlingar.\\

\noindent\textbf{2017-10-29}\\
\noindent\rule{\textwidth}{1pt}

\noindent 
Idag lade jag till så att parsern klarar av elseif och else satser, samt externa
funktionsanrop. Dessutom lade jag till semantiska kontroller för 
main-funktionen.\\

\noindent\textbf{2017-11-01}\\
\noindent\rule{\textwidth}{1pt}

\noindent 
Nu har jag lagt till experimentell kodgenerering. Jag var tvungen att själv 
skapa en abstraktion av C-stränggränssnittet eftersom att det inte var 
tillräckligt användbart för mina ändamål. \\

\noindent\textbf{2017-11-02}\\
\noindent\rule{\textwidth}{1pt}

\noindent 
Kompilatorn stödjer nu argument för att stödja frivillig utskrivning av AST och 
tokens med mera. C har inget sådant bibliotek i standarden, så jag fick skriva 
en API för \textit{POSIX}-baserade kommandoradsargument.\\

\noindent\textbf{2017-11-03}\\
\noindent\rule{\textwidth}{1pt}

\noindent 
Kompilatorn stödjer nu automatisk kompilering av den genererade C koden med 
hjälp av C kompilatorn \textit{gcc}. \\

\noindent\textbf{2017-11-04}\\
\noindent\rule{\textwidth}{1pt}

\noindent 
Parsern och kodgeneratorn stödjer nu tomma typer. Generatorn skapar nu korrekt 
indenterad C kod, och genererar if-satser korrekt. Semantiska tester för externa
funktionsanrop samt memory leaks i semantiska tester är nu fixade. \\

\noindent\textbf{2017-11-05}\\
\noindent\rule{\textwidth}{1pt}

\noindent 
Små förbättringar av funktionsgenereringen lades till.\\

\noindent\textbf{2017-11-07}\\
\noindent\rule{\textwidth}{1pt}

\noindent 
Idag lade jag till generering av uttryck samt extra semantisk information in i 
AST noderna. Experimentell generering av lokala funktioner och funktionsanrop 
lades även till. \\

\noindent\textbf{2017-11-08}\\
\noindent\rule{\textwidth}{1pt}

\noindent 
Idag fixade jag en bug som gjorde att lokala variabler inte genererades korrekt.
\\

\noindent\textbf{2017-11-26}\\
\noindent\rule{\textwidth}{1pt}

\noindent 
Jag har nu arbetat i mer än två veckor med att omstrukturera hela 
projektet. Anledningen var att felrapporteringen inte var tillräckligt bra, 
samt att mycket blev rörigt. Det var bättre att strukurera om helt än att ändra 
på allt som redan var skrivet bedömde jag. Hela projektet är nu betydligt bättre
strukturerat vilket gör att ändringar blir enklare. Felrapporteringen är nu 
tydlig och färgkodad. Semantiska analyser för oanvända och oinitierade variabler
fungerar nu korrekt, och parsern stödjer nu tomma return-statements. \\

\noindent\textbf{2017-12-01}\\
\noindent\rule{\textwidth}{1pt}

\noindent 
Semantisk analys för att se om funktioner returnerar korrekt fungerar nu. Jag 
fixade även lite memory leaks som jag upptäckte. Jag började även med 
enkel optimering, men det visade sig vara betydligt svårare än jag trodde.\\

\noindent\textbf{2017-12-21}\\
\noindent\rule{\textwidth}{1pt}

\noindent 
Under veckan har jag testat att implementera en enkelt interpretator för 
compile-time execution av kod. Efter ett tag bedömde jag dock att detta var
ett för stort ämne att börja på nu. \\

\noindent\textbf{2017-12-29}\\
\noindent\rule{\textwidth}{1pt}

\noindent 
Jag har lagt till små förbättringar till generatorn och analysatorn. Bland annat 
klarar kodgeneratorn nu externa funktionsanrop, if-satser samt typkonversioner. 
\\

\noindent\textbf{2018-01-01}\\
\noindent\rule{\textwidth}{1pt}

\noindent 
Återigen har jag fixad kodgeneratorn då den fortfarande hade vissa problem. 
Lokala funktioner samt exponenter i uttryck fungerar nu som de ska göra.\\

\noindent\textbf{2018-01-05}\\
\noindent\rule{\textwidth}{1pt}

\noindent 
Idag gjorde jag så att parsern och generatorn hanterar defer-statements. \\

\noindent\textbf{2018-01-06}\\
\noindent\rule{\textwidth}{1pt}

\noindent 
While-loopar parse:as och generaras nu korrekt. Jag lade också till 
förbättringar till felmeddelanden.\\

\noindent\textbf{2018-01-20}\\
\noindent\rule{\textwidth}{1pt}

\noindent 
Prestanda-tester har nu lagts till, så att man enkelt kan se hur lång tid varje
komponent i kompilatorn tar. Fixade även några små fel som jag hittade.\\

\noindent\textbf{2018-01-21}\\
\noindent\rule{\textwidth}{1pt}

\noindent 
Ett riktigt typsystem har nu blivit implementerat. Det kan hantera listor, 
pekare, funktioner, unioner, enumerationer structurer samt typdefinitioner.\\

\subsection{Resultatet}

Kompilatorn och språket är långt ifrån klart. Många av specifikationerna som 
sattes i början av projektet han inte bli färdiga. Dock så är Erwall nu ett 
fungerande programmeringsspråk som jag bland annat använde en matematiklektion,
vilket tyder på att det redan är användbart och på vissa sätt till och med 
bättre än C. Bland annat är typsystemet betydligt mer avancerat, och konstanter
samt sematiska kontroller och felmeddelande är betydligt bättre i jämförelse. I 
teorin skulle man kunna skriva vilka program som helst i det. 

\section{Diskussion och slutsatser}
This section's content...

\section{Källförteckning}
This section's content...

\section{Bilagor}
This section's content...

\end{document}
